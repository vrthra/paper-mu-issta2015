\section{Results}
\label{sec:results}
We see from the data that random sampling has a higher $R^2$, and also approximates the all  mutations score better.
However, caveat emptor. There is a difference between the number of mutators for SDL and RAND. If our approximation was right, we should have gotten nearly same number of mutators. However
The number of mutators in RND is
 Sexpr{sum(m.lrnd.pit.mutation.total) / sum(m.sdl.pit.mutation.total)} times SDL.

The total mutants for all is Sexpr{sum(m.pit.mutation.total)}, for rnd is Sexpr{sum(m.lrnd.pit.mutation.total)} and for sdl is Sexpr{sum(m.sdl.pit.mutation.total)}

\begin{enumerate}
\item [x] For op-selection to work, the distributions of mutations across different projects has to be similar enough. This can be checked by t-test.
\item [TODO] We should check if the distribution significantly different too.
\item [DONE] Use sampling
\item [DONE] Check on randoop samples too
\item [NO] Implement bomb (athrow) and see its mutation score, and see how close it comes.
\item [DONE] Compare random selection with class, method, and mutagen sampling.
\item [TODO] Report time for each, as cost.
\item [TODOX] Find the mutant count to LOC multiplier, and use it to get the probability of choosing for each project.
\end{enumerate}