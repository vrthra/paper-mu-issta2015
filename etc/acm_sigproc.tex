\batchmode
% THIS IS SIGPROC-SP.TEX - VERSION 3.1
% WORKS WITH V3.2SP OF ACM_PROC_ARTICLE-SP.CLS
% APRIL 2009
%
% It is an example file showing how to use the 'acm_proc_article-sp.cls' V3.2SP
% LaTeX2e document class file for Conference Proceedings submissions.
% ----------------------------------------------------------------------------------------------------------------
% This .tex file (and associated .cls V3.2SP) *DOES NOT* produce:
%       1) The Permission Statement
%       2) The Conference (location) Info information
%       3) The Copyright Line with ACM data
%       4) Page numbering
% ---------------------------------------------------------------------------------------------------------------
% It is an example which *does* use the .bib file (from which the .bbl file
% is produced).
% REMEMBER HOWEVER: After having produced the .bbl file,
% and prior to final submission,
% you need to 'insert'  your .bbl file into your source .tex file so as to provide
% ONE 'self-contained' source file.
%
% Questions regarding SIGS should be sent to
% Adrienne Griscti ---> griscti@acm.org
%
% Questions/suggestions regarding the guidelines, .tex and .cls files, etc. to
% Gerald Murray ---> murray@hq.acm.org
%
% For tracking purposes - this is V3.1SP - APRIL 2009

\documentclass{sig-alternate}
\usepackage{float}
\usepackage{Sweavel}
\usepackage{placeins}
\usepackage{comment}
\usepackage[bookmarks=true, urlcolor=blue, linkcolor=blue, citecolor=blue, colorlinks=true, pdftitle={A comparison of mutant reduction approaches}, pdfauthor={Rahul Gopinath}]{hyperref}
\usepackage[center]{caption}

\usepackage[font=footnotesize]{subfig}

% IMPORTANT : Remove this if you get an error about copyright box
\DeclareCaptionType{copyrightbox}

\setcounter{topnumber}{2}
\setcounter{bottomnumber}{2}
\setcounter{totalnumber}{4}
\renewcommand{\topfraction}{0.85}
\renewcommand{\bottomfraction}{0.85}
\renewcommand{\textfraction}{0.15}
\renewcommand{\floatpagefraction}{0.7}

\newcommand{\ignore}[1]{}
 %------------begin Float Adjustment
%two column float page must be 90% full
%\renewcommand\dblfloatpagefraction{.90}
%two column top float can cover up to 80% of page
%\renewcommand\dbltopfraction{.80}
%float page must be 90% full
%\renewcommand\floatpagefraction{.90}
%top float can cover up to 80% of page
%\renewcommand\topfraction{.80}
%bottom float can cover up to 80% of page
%\renewcommand\bottomfraction{.80}
%at least 10% of a normal page must contain text
%\renewcommand\textfraction{.1}
%separation between floats and text
\setlength\dbltextfloatsep{9pt plus 5pt minus 3pt }
%separation between two column floats and text
\setlength\textfloatsep{4pt plus 2pt minus 1.5pt}

\begin{document}

\title{How Good is Statement Deletion?}
\subtitle{A Comparison Between Statement Deletion and Random Sampling}


%
% You need the command \numberofauthors to handle the 'placement
% and alignment' of the authors beneath the title.
%
% For aesthetic reasons, we recommend 'three authors at a time'
% i.e. three 'name/affiliation blocks' be placed beneath the title.
%
% NOTE: You are NOT restricted in how many 'rows' of
% "name/affiliations" may appear. We just ask that you restrict
% the number of 'columns' to three.
%
% Because of the available 'opening page real-estate'
% we ask you to refrain from putting more than six authors
% (two rows with three columns) beneath the article title.
% More than six makes the first-page appear very cluttered indeed.
%
% Use the \alignauthor commands to handle the names
% and affiliations for an 'aesthetic maximum' of six authors.
% Add names, affiliations, addresses for
% the seventh etc. author(s) as the argument for the
% \additionalauthors command.
% These 'additional authors' will be output/set for you
% without further effort on your part as the last section in
% the body of your article BEFORE References or any Appendices.

\numberofauthors{4}
\author{
Rahul Gopinath\\
       \affaddr{Oregon State University}\\
       \email{gopinath@eecs.orst.edu}
\alignauthor
Amin Alipour\\
       \affaddr{Oregon State University}\\
       \email{alipour@eecs.orst.edu}
\and
Carlos Jensen\\
       \affaddr{Oregon State University}\\
       \email{cjensen@eecs.orst.edu}
\alignauthor
Alex Groce\\
       \affaddr{Oregon State University}\\
       \email{agroce@gmail.com}
}



\maketitle
\begin{abstract}
% 1.Statetheproblem
Mutation analysis is a well-known method for measuring the quality of
test suites. However, it is computationally intensive compared to other
measures, which makes it hard to use in practice. Multiple strategies have
been proposed to reduce the number of mutants produced, and hence reduce
the computational requirements.

%
%2.Saywhyit’saninterestingproblem
However, current research on mutation reduction is beset with several
problems. There is inconclusive evidence supporting many of the suggested
reduction strategies due to the limited number and size of subject programs
investigated. The lack of a comprehensive study comparing even major approaches
is another problem. Finally the traditional evaluation criteria for mutation
reduction approaches relies on mutation-adequate test suites which rarely occur
in real world. There is little evidence that the strategies that perform well
using the traditional evaluation crtiera leads to better mutation selection
for non-adequate tests in the real world.

%3.Saywhatyoursolutionachieves

We propose a noval evaluation criteria of reduction strategies that is
usable with non-mutation adequate test suites, and link it to the actual
use of mutation analysis during development -- to produce tests that check
for as many different bugs as possible.

We evaluate the performance of various selection strategies on 312 opensource
projects from github, which shows that the popular reduction strategies:
operator selection, stratified sampling of mutations based on operators,
and stratified sampling of mutations based on program elements are hardly
better than random sampling, and often performs worse.  This suggests a need
for further research on better methods of mutation reduction strategies.

%4.Saywhatfollowsfromyoursolution

% Technical papers must be prepared in ACM conference format and must not exceed 10 pages
% (including figures and appendices but NOT including references). That is, any pages after
% the tenth must contain only references. All submissions must be in English. Submit your
% paper via the paper submission website. Submissions that do not adhere to these
% guidelines or that violate formatting will be declined without review.

\end{abstract}

%This is from the 98 class, 2012 does not have the number.
\category{D.2.5}{Software Engineering}{Testing and Debugging}{ Testing Tools}

\terms{Measurement, Verification}

\keywords{Test frameworks, empirical analysis, mutation operators} % NOT required for Proceedings


\section{Introduction}

Here is a sample cite \cite{bib2013sample}. Use Math as normal
$1 + 2 = 3$

\subsection{Heading 1}

Some Lists

\begin{itemize}
\item
  \emph{L1} Some info
\item
  \emph{L2} Some more info
\end{itemize}

\subsubsection{H3}

You can use \emph{italics} and \textbf{bold}

\paragraph{H4}

 A simple table here: This is shown in
Table\textasciitilde{}\ref{tab:sample1} as below.

\ctable[caption = {A random table\label{tab:sample1}},
pos = H, center, botcap]{llll}
{% notes
}
{% rows
\FL
Types & Active & Passive & Medium
\ML
People & 64\% & 63\% & 60\%
\\\noalign{\medskip}
Cats & 35\% & 31\% & 36\%
\LL
}

More text.




\lipsum[1]
\begin{Schunk}
\begin{Sinput}
render_listings()
\end{Sinput}
\end{Schunk}


\begin{figure*}
\begin{Schunk}
\begin{Sinput}
x = rnorm(100)
par(mar = c(4, 4, 0.1, 0.1))
boxplot(x)
hist(x, main = "")
\end{Sinput}


{\centering \includegraphics[width=.45\linewidth]{figure/twocolumn-boring-plots1} \includegraphics[width=.45\linewidth]{figure/twocolumn-boring-plots2} 

}

\end{Schunk}


\caption{Two plots spanning over two columns.}
\end{figure*}
\lipsum[1]

\section{Conclusion}

A rather simple template to convert markdown to ieee paper pdf.

%
% The following two commands are all you need in the
% initial runs of your .tex file to
% produce the bibliography for the citations in your paper.
\bibliographystyle{abbrv}
\bibliography{paper}  % sigproc.bib is the name of the Bibliography in this case
\end{document}
