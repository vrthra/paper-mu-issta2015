% 1.Statetheproblem
Mutation analysis is a metric for measuring the quality of test suites. However, its exorbitant computational requirements makes it hard to use in practice. Choosing a smaller subset of mutations to run is a simple approach that can alleviate this problem. There exist multiple ways to reduce the set of mutations to run. Operator selection has recently been in the focus. However recently researchers have found that sampling can achive accuracy and mutant reduction similar to operator selection.

%2.Saywhyit’saninterestingproblem
However, which sampling technique is best, is an open problem, especially for real-world programs and test-suites.

%3.Saywhatyoursolutionachieves
Our research compares a number of sampling criteria, and operator selection criteria based on their ability to approximate the full mutation score, and also the consistency of results.

%4.Saywhatfollowsfromyoursolution
Our experiments rank sampling criterias that can be used by testers according how much mutant reduction is achieved, their ability to predict the final mutation score, the relative effect of unit increase in sampled mutation score on the final score, and also the time required for the run. We also find that sampling techniques work better than operator selection methods including statement deletion in both mutation reduction and accuracy of approximation to the full mutation score.

% Technical papers must be prepared in ACM conference format and must not exceed 10 pages
% (including figures and appendices but NOT including references). That is, any pages after
% the tenth must contain only references. All submissions must be in English. Submit your
% paper via the paper submission website. Submissions that do not adhere to these
% guidelines or that violate formatting will be declined without review.
