% 1.Statetheproblem
Mutation analysis is a well-known method for measuring the quality of
test suites. However, it is computationally intensive compared to other
measures, which makes it hard to use in practice. Multiple strategies have
been proposed to reduce the number of mutants produced, and hence reduce
the computational requirements.

%
%2.Saywhyit’saninterestingproblem
However, current research on mutation reduction is beset with several
problems. There is inconclusive evidence supporting many of the suggested
reduction strategies due to the limited number and size of subject programs
investigated. The lack of a comprehensive study comparing even major approaches
is another problem. Finally the traditional evaluation criteria for mutation
reduction approaches relies on mutation-adequate test suites which rarely occur
in real world. There is little evidence that the strategies that perform well
using the traditional evaluation criteria leads to better mutation selection
for non-adequate tests in the real world.

%3.Saywhatyoursolutionachieves

We propose a novel evaluation criteria of reduction strategies that is
usable with non-mutation adequate test suites, and link it to the actual
use of mutation analysis during development -- to produce tests that check
for as many different bugs as possible.

We evaluate the performance of various selection strategies on 312 opensource
projects from Github, which shows that the popular reduction strategies:
operator selection, stratified sampling of mutations based on operators,
and stratified sampling of mutations based on program elements are hardly
better than random sampling, and often performs worse.  This suggests a need
for further research on better methods of mutation reduction strategies.

%4.Saywhatfollowsfromyoursolution

% Technical papers must be prepared in ACM conference format and must not exceed 10 pages
% (including figures and appendices but NOT including references). That is, any pages after
% the tenth must contain only references. All submissions must be in English. Submit your
% paper via the paper submission website. Submissions that do not adhere to these
% guidelines or that violate formatting will be declined without review.
