% 1.Statetheproblem
Mutation analysis is a metric for measuring the quality of test
suites. However, it is computationally intensive compared to other measures
which makes it hard to use in practice. Choosing a smaller subset of mutations
to run is a simple approach that can alleviate this problem.
Operator selection is an area that has been heavily researched. Recently,
researchers have found that sampling can achieve accuracy and mutant
reduction similar to operator selection.

%2.Saywhyit’saninterestingproblem
However, the statistical inference that can be drawn from the previous
research has been limited due to the small number of subject programs. Further,
which sampling technique is best is still an open problem that needs to be
tackled on a much larger range of test subjects.

%3.Saywhatyoursolutionachieves
Our research compares a number of sampling and operator selection criteria
based on their ability to approximate the full mutation score, and also
the consistency of mutation reduction ratios achieved. Our results provide
a ranking that can be used to choose an appropriate mutation reduction
technique according to the mutation reduction ratio desired, and how much
reduction in accuracy is acceptable.

%4.Saywhatfollowsfromyoursolution
We find that all sampling criterias perform better than operator selection
methods when considering their ability to predict the full mutation score,
and also the consistency of mutation reduction ratios achieved.  Specifically,
we find that random selection of a mutant performs better than statement
deletion operator selection criteria.

% Technical papers must be prepared in ACM conference format and must not exceed 10 pages
% (including figures and appendices but NOT including references). That is, any pages after
% the tenth must contain only references. All submissions must be in English. Submit your
% paper via the paper submission website. Submissions that do not adhere to these
% guidelines or that violate formatting will be declined without review.
